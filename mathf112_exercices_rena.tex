\documentclass{report}
\usepackage[latin1]{inputenc}
\usepackage[T1]{fontenc}
\usepackage[francais]{babel}
\usepackage[top=1.5cm, bottom=2cm, left=3cm, right=3cm]{geometry}
\usepackage{graphicx}
\usepackage{hyperref}
\usepackage{amsmath}
\usepackage{enumerate}
\usepackage{multicol} % multi columns
\usepackage{color}
\usepackage{colortbl}
\setcounter{tocdepth}{0} % only chapters appear in 'table of contents'


  \title{\Large{\bsc{MATH-F-112 - Math\'{e}matiques}}\\
  Exercices - Module A
  }

  \author{Renato \bsc{Costa Ribeiro}}
  \date{21/09/2015}


\begin{document}
\maketitle
\tableofcontents

%%%%%%%%%%%%%%%%%%%%%%%%%%%%%%%%%%%%%%%%%%%%%CHAPITRE 1%%%%%%%%%%%%%%%%%%%%%%%%%%%%%%%%%%%%%%%%%%%%%%%%%%%%%
\chapter{Logique}
  \section{} % 1.1
    \begin{multicols}{4}
      \begin{enumerate}[a.]
	\item $B \Rightarrow G$
	\item $R \Rightarrow S$
	\item $(P \vee N) \Rightarrow A$
	\item $H \Rightarrow \neg V$
	\item $O \Rightarrow P$
	\item $(O \Rightarrow N) \wedge (N \Rightarrow O)$
	\item $D \Rightarrow C$
	\item $\neg C \Rightarrow \neg I$
	\item $D \Rightarrow C$
	\item $E \Rightarrow N$
	\item $(T \wedge N) \Rightarrow I$
	\item $I \Rightarrow P$
	\item $V \Rightarrow D$
      \end{enumerate}
    \end{multicols}

    
  \section{} % 1.2

  \begin{enumerate}[(1).]
  \item $A \vee B \vee C$
  \item $C \Rightarrow A$
  \item $B \Rightarrow (A \vee C)$
  \end{enumerate}
  Pour savoir si A est le coupable il faut: $(1)\wedge(2)\wedge(3)$


\definecolor{tcA}{rgb}{0,0,0}
\begin{center}
\begin{tabular}{ccccccccc}
% use packages: color,colortbl
 &  &  & $A \vee B \vee C$ & $C \Rightarrow A$ &  & $B \Rightarrow (A \vee C)$ &  & \\
\textbf{\textcolor{tcA}{$\textbf{A}$}} & \textbf{\textcolor{tcA}{$\textbf{B}$}} & \textbf{\textcolor{tcA}{$\textbf{C}$}} & \textbf{\textcolor{tcA}{(1)}} & \textbf{\textcolor{tcA}{(2)}} & \textbf{\textcolor{tcA}{$A \vee C$}} & \textbf{\textcolor{tcA}{(3)}} & \textbf{\textcolor{tcA}{$(1)\wedge(2)\wedge(3)$}} & \textbf{\textcolor{tcA}{$(1)\wedge(2)\wedge(3) \Leftrightarrow A $}}\\
0 & 0 & 0 & 0 & 1 & 0 & 1 & 0 & 1\\
0 & 0 & 1 & 1 & 0 & 1 & 1 & 0 & 1\\
0 & 1 & 0 & 1 & 1 & 0 & 0 & 0 & 1\\
0 & 1 & 1 & 1 & 0 & 1 & 1 & 0 & 1\\
1 & 0 & 0 & 1 & 1 & 1 & 1 & 1 & 1\\
1 & 0 & 1 & 1 & 1 & 1 & 1 & 1 & 1\\
1 & 1 & 0 & 1 & 1 & 1 & 1 & 1 & 1\\
1 & 1 & 1 & 1 & 1 & 1 & 1 & 1 & 1
\end{tabular}
\end{center}

  
Comme nous pouvons le constater, la derni\`{e}re colonne prouve que $(1)\wedge(2)\wedge(3) \Leftrightarrow A $. A est donc le coupable.
 
 
 
 
 
 
 
 
 

\end{document}